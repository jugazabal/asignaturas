% Options for packages loaded elsewhere
% Options for packages loaded elsewhere
\PassOptionsToPackage{unicode}{hyperref}
\PassOptionsToPackage{hyphens}{url}
\PassOptionsToPackage{dvipsnames,svgnames,x11names}{xcolor}
%
\documentclass[
  spanish,
  12pt,
  a4paper,
]{article}
\usepackage{xcolor}
\usepackage[top=2.5cm,bottom=2.5cm,left=2.5cm,right=2.5cm]{geometry}
\usepackage{amsmath,amssymb}
\setcounter{secnumdepth}{5}
\usepackage{iftex}
\ifPDFTeX
  \usepackage[T1]{fontenc}
  \usepackage[utf8]{inputenc}
  \usepackage{textcomp} % provide euro and other symbols
\else % if luatex or xetex
  \usepackage{unicode-math} % this also loads fontspec
  \defaultfontfeatures{Scale=MatchLowercase}
  \defaultfontfeatures[\rmfamily]{Ligatures=TeX,Scale=1}
\fi
\usepackage{lmodern}
\ifPDFTeX\else
  % xetex/luatex font selection
\fi
% Use upquote if available, for straight quotes in verbatim environments
\IfFileExists{upquote.sty}{\usepackage{upquote}}{}
\IfFileExists{microtype.sty}{% use microtype if available
  \usepackage[]{microtype}
  \UseMicrotypeSet[protrusion]{basicmath} % disable protrusion for tt fonts
}{}
\usepackage{setspace}
\makeatletter
\@ifundefined{KOMAClassName}{% if non-KOMA class
  \IfFileExists{parskip.sty}{%
    \usepackage{parskip}
  }{% else
    \setlength{\parindent}{0pt}
    \setlength{\parskip}{6pt plus 2pt minus 1pt}}
}{% if KOMA class
  \KOMAoptions{parskip=half}}
\makeatother
% Make \paragraph and \subparagraph free-standing
\makeatletter
\ifx\paragraph\undefined\else
  \let\oldparagraph\paragraph
  \renewcommand{\paragraph}{
    \@ifstar
      \xxxParagraphStar
      \xxxParagraphNoStar
  }
  \newcommand{\xxxParagraphStar}[1]{\oldparagraph*{#1}\mbox{}}
  \newcommand{\xxxParagraphNoStar}[1]{\oldparagraph{#1}\mbox{}}
\fi
\ifx\subparagraph\undefined\else
  \let\oldsubparagraph\subparagraph
  \renewcommand{\subparagraph}{
    \@ifstar
      \xxxSubParagraphStar
      \xxxSubParagraphNoStar
  }
  \newcommand{\xxxSubParagraphStar}[1]{\oldsubparagraph*{#1}\mbox{}}
  \newcommand{\xxxSubParagraphNoStar}[1]{\oldsubparagraph{#1}\mbox{}}
\fi
\makeatother


\usepackage{longtable,booktabs,array}
\usepackage{calc} % for calculating minipage widths
% Correct order of tables after \paragraph or \subparagraph
\usepackage{etoolbox}
\makeatletter
\patchcmd\longtable{\par}{\if@noskipsec\mbox{}\fi\par}{}{}
\makeatother
% Allow footnotes in longtable head/foot
\IfFileExists{footnotehyper.sty}{\usepackage{footnotehyper}}{\usepackage{footnote}}
\makesavenoteenv{longtable}
\usepackage{graphicx}
\makeatletter
\newsavebox\pandoc@box
\newcommand*\pandocbounded[1]{% scales image to fit in text height/width
  \sbox\pandoc@box{#1}%
  \Gscale@div\@tempa{\textheight}{\dimexpr\ht\pandoc@box+\dp\pandoc@box\relax}%
  \Gscale@div\@tempb{\linewidth}{\wd\pandoc@box}%
  \ifdim\@tempb\p@<\@tempa\p@\let\@tempa\@tempb\fi% select the smaller of both
  \ifdim\@tempa\p@<\p@\scalebox{\@tempa}{\usebox\pandoc@box}%
  \else\usebox{\pandoc@box}%
  \fi%
}
% Set default figure placement to htbp
\def\fps@figure{htbp}
\makeatother


% definitions for citeproc citations
\NewDocumentCommand\citeproctext{}{}
\NewDocumentCommand\citeproc{mm}{%
  \begingroup\def\citeproctext{#2}\cite{#1}\endgroup}
\makeatletter
 % allow citations to break across lines
 \let\@cite@ofmt\@firstofone
 % avoid brackets around text for \cite:
 \def\@biblabel#1{}
 \def\@cite#1#2{{#1\if@tempswa , #2\fi}}
\makeatother
\newlength{\cslhangindent}
\setlength{\cslhangindent}{1.5em}
\newlength{\csllabelwidth}
\setlength{\csllabelwidth}{3em}
\newenvironment{CSLReferences}[2] % #1 hanging-indent, #2 entry-spacing
 {\begin{list}{}{%
  \setlength{\itemindent}{0pt}
  \setlength{\leftmargin}{0pt}
  \setlength{\parsep}{0pt}
  % turn on hanging indent if param 1 is 1
  \ifodd #1
   \setlength{\leftmargin}{\cslhangindent}
   \setlength{\itemindent}{-1\cslhangindent}
  \fi
  % set entry spacing
  \setlength{\itemsep}{#2\baselineskip}}}
 {\end{list}}
\usepackage{calc}
\newcommand{\CSLBlock}[1]{\hfill\break\parbox[t]{\linewidth}{\strut\ignorespaces#1\strut}}
\newcommand{\CSLLeftMargin}[1]{\parbox[t]{\csllabelwidth}{\strut#1\strut}}
\newcommand{\CSLRightInline}[1]{\parbox[t]{\linewidth - \csllabelwidth}{\strut#1\strut}}
\newcommand{\CSLIndent}[1]{\hspace{\cslhangindent}#1}

\ifLuaTeX
\usepackage[bidi=basic]{babel}
\else
\usepackage[bidi=default]{babel}
\fi
% get rid of language-specific shorthands (see #6817):
\let\LanguageShortHands\languageshorthands
\def\languageshorthands#1{}


\setlength{\emergencystretch}{3em} % prevent overfull lines

\providecommand{\tightlist}{%
  \setlength{\itemsep}{0pt}\setlength{\parskip}{0pt}}



 


\makeatletter
\@ifpackageloaded{tcolorbox}{}{\usepackage[skins,breakable]{tcolorbox}}
\@ifpackageloaded{fontawesome5}{}{\usepackage{fontawesome5}}
\definecolor{quarto-callout-color}{HTML}{909090}
\definecolor{quarto-callout-note-color}{HTML}{0758E5}
\definecolor{quarto-callout-important-color}{HTML}{CC1914}
\definecolor{quarto-callout-warning-color}{HTML}{EB9113}
\definecolor{quarto-callout-tip-color}{HTML}{00A047}
\definecolor{quarto-callout-caution-color}{HTML}{FC5300}
\definecolor{quarto-callout-color-frame}{HTML}{acacac}
\definecolor{quarto-callout-note-color-frame}{HTML}{4582ec}
\definecolor{quarto-callout-important-color-frame}{HTML}{d9534f}
\definecolor{quarto-callout-warning-color-frame}{HTML}{f0ad4e}
\definecolor{quarto-callout-tip-color-frame}{HTML}{02b875}
\definecolor{quarto-callout-caution-color-frame}{HTML}{fd7e14}
\makeatother
\makeatletter
\@ifpackageloaded{caption}{}{\usepackage{caption}}
\AtBeginDocument{%
\ifdefined\contentsname
  \renewcommand*\contentsname{Tabla de contenidos}
\else
  \newcommand\contentsname{Tabla de contenidos}
\fi
\ifdefined\listfigurename
  \renewcommand*\listfigurename{Listado de Figuras}
\else
  \newcommand\listfigurename{Listado de Figuras}
\fi
\ifdefined\listtablename
  \renewcommand*\listtablename{Listado de Tablas}
\else
  \newcommand\listtablename{Listado de Tablas}
\fi
\ifdefined\figurename
  \renewcommand*\figurename{Figura}
\else
  \newcommand\figurename{Figura}
\fi
\ifdefined\tablename
  \renewcommand*\tablename{Tabla}
\else
  \newcommand\tablename{Tabla}
\fi
}
\@ifpackageloaded{float}{}{\usepackage{float}}
\floatstyle{ruled}
\@ifundefined{c@chapter}{\newfloat{codelisting}{h}{lop}}{\newfloat{codelisting}{h}{lop}[chapter]}
\floatname{codelisting}{Listado}
\newcommand*\listoflistings{\listof{codelisting}{Listado de Listados}}
\makeatother
\makeatletter
\makeatother
\makeatletter
\@ifpackageloaded{caption}{}{\usepackage{caption}}
\@ifpackageloaded{subcaption}{}{\usepackage{subcaption}}
\makeatother
\usepackage{bookmark}
\IfFileExists{xurl.sty}{\usepackage{xurl}}{} % add URL line breaks if available
\urlstyle{same}
\hypersetup{
  pdftitle={Actividad 3: Plan de Orientación Académico-Profesional (Aula del Futuro)},
  pdfauthor={Juan Marcos García Aranzábal},
  pdflang={es},
  colorlinks=true,
  linkcolor={blue},
  filecolor={Maroon},
  citecolor={Blue},
  urlcolor={Blue},
  pdfcreator={LaTeX via pandoc}}


\title{Actividad 3: Plan de Orientación Académico-Profesional (Aula del
Futuro)}
\author{Juan Marcos García Aranzábal}
\date{2026-01-15}
\begin{document}
\maketitle

\renewcommand*\contentsname{Tabla de contenidos}
{
\hypersetup{linkcolor=}
\setcounter{tocdepth}{3}
\tableofcontents
}

\setstretch{1.15}
\section{Portada}\label{portada}

\textbf{Máster Universitario en Formación del Profesorado de ESO y
Bachillerato, FP y Enseñanza de Idiomas}\\
\textbf{Especialidad:} Orientación Educativa\\
\textbf{Asignatura:} Practicum\\
\textbf{Actividad 3:} Plan de Orientación Académico-Profesional
(Integración Aula del Futuro)\\
\textbf{Programa analizado:} Aula del Futuro (AdF) -- INTEF (en marcha
desde 2021--2022, coordinación local: María del Carmen Mateos)\\
\textbf{Centro de prácticas:} IESO Quercus (Malpartida de Plasencia,
4673 hab., \textasciitilde180 alumnos)\\
\textbf{Curso destinatario:} 4.º ESO\\
\textbf{Ventana de aplicación simulada:} 12/01/2026 -- 08/03/2026\\
\textbf{Estudiante:} Juan Marcos García Aranzábal\\
\textbf{Fecha de entrega:} 15/01/2026\\
\textbf{CCAA de referencia:} Extremadura

\begin{tcolorbox}[enhanced jigsaw, colframe=quarto-callout-note-color-frame, opacityback=0, opacitybacktitle=0.6, breakable, coltitle=black, colbacktitle=quarto-callout-note-color!10!white, left=2mm, rightrule=.15mm, colback=white, bottomtitle=1mm, toprule=.15mm, bottomrule=.15mm, arc=.35mm, toptitle=1mm, leftrule=.75mm, title=\textcolor{quarto-callout-note-color}{\faInfo}\hspace{0.5em}{Nota}, titlerule=0mm]

Esta actividad sintetiza un plan modular de orientación
académico-profesional apoyado en metodologías activas y escenarios
flexibles del Aula del Futuro, alineado con modelos de desarrollo de
carrera (Holland, Super, Krumboltz, Savickas) y necesidades decisionales
de 4.º ESO.

\end{tcolorbox}

\section{PARTE 1. Revisión de recursos (máx. 1
página)}\label{parte-1.-revisiuxf3n-de-recursos-muxe1x.-1-puxe1gina}

Tabla resumen (8 recursos seleccionados). Cada entrada recoge criterios
de pertinencia para orientación vocacional en contexto semi-rural.

\begin{longtable}[]{@{}
  >{\raggedright\arraybackslash}p{(\linewidth - 16\tabcolsep) * \real{0.1111}}
  >{\raggedright\arraybackslash}p{(\linewidth - 16\tabcolsep) * \real{0.1111}}
  >{\raggedright\arraybackslash}p{(\linewidth - 16\tabcolsep) * \real{0.1111}}
  >{\raggedright\arraybackslash}p{(\linewidth - 16\tabcolsep) * \real{0.1111}}
  >{\raggedright\arraybackslash}p{(\linewidth - 16\tabcolsep) * \real{0.1111}}
  >{\raggedright\arraybackslash}p{(\linewidth - 16\tabcolsep) * \real{0.1111}}
  >{\raggedright\arraybackslash}p{(\linewidth - 16\tabcolsep) * \real{0.1111}}
  >{\raggedright\arraybackslash}p{(\linewidth - 16\tabcolsep) * \real{0.1111}}
  >{\raggedright\arraybackslash}p{(\linewidth - 16\tabcolsep) * \real{0.1111}}@{}}
\toprule\noalign{}
\begin{minipage}[b]{\linewidth}\raggedright
Recurso
\end{minipage} & \begin{minipage}[b]{\linewidth}\raggedright
Autor/Entidad
\end{minipage} & \begin{minipage}[b]{\linewidth}\raggedright
Año (estim.)
\end{minipage} & \begin{minipage}[b]{\linewidth}\raggedright
Objetivos clave
\end{minipage} & \begin{minipage}[b]{\linewidth}\raggedright
Destinatarios
\end{minipage} & \begin{minipage}[b]{\linewidth}\raggedright
Contenidos
\end{minipage} & \begin{minipage}[b]{\linewidth}\raggedright
Formato
\end{minipage} & \begin{minipage}[b]{\linewidth}\raggedright
Acceso
\end{minipage} & \begin{minipage}[b]{\linewidth}\raggedright
URL
\end{minipage} \\
\midrule\noalign{}
\endhead
\bottomrule\noalign{}
\endlastfoot
Aula del Futuro (AdF) & INTEF & 2021--2025 & Integrar metodologías
activas y escenarios para exploración vocacional & ESO/Bachillerato &
Espacios, escenarios, personalización & Mixto (espacio + guías) &
Gratuito & https://auladelfuturo.educalab.es/ \\
Future Classroom Lab (FCL) \& Toolkit & European Schoolnet & 2025 &
Diseño de escenarios y competencias digitales & Docentes / Estudiantes
12--18 & Diseño, innovación, digital competence & Digital & Gratuito &
http://fcl.eun.org/toolkit \\
Educarex -- Innovación/Orientación & Consejería Educación Extremadura &
2025 & Recursos regionales orientación y FP & ESO / FP inicial &
Itinerarios, materiales CCAA & Web & Gratuito &
https://www.educarex.es/ \\
TodoFP & Ministerio Educación FP y Deportes & 2025 & Información oficial
de familias y grados FP & Alumnado que decide itinerario & Itinerarios,
familias, estándares competencia & Web & Gratuito &
https://www.todofp.es/ \\
SEPE -- Observatorio Ocupaciones & SEPE & 2025 & Datos mercado laboral y
perfiles emergentes & 4.º ESO / Bachillerato & Sectores, tendencias,
empleabilidad & Web / Informes PDF & Gratuito & https://sepe.es/ \\
Mi Vocación & Fundación Bertelsmann & 2025 & Autoconocimiento y
exploración vocacional & 14--18 años & Intereses, perfiles profesionales
& Web interactiva & Registro gratuito & https://www.mivocacion.org/ \\
Descubre la FP & F. Atresmedia \& F. Mapfre & 2025 & Divulgación
atractiva de la FP & ESO / Transición FP & Familias FP, testimonios &
Web / Multimedia & Gratuito & https://descubrelafp.org/ \\
FCL Courses Catalogue & European Schoolnet & 2025 & Formación docente
para innovación que apoya orientación & Docentes & Active learning,
digital, career links & Web / Cursos & Mixto (algunos de pago) &
https://fcl.eun.org/courses \\
\end{longtable}

\textbf{Criterios de selección:} Relevancia para planificación de
carrera, accesibilidad gratuita, actualización, potencial de integración
con escenarios AdF, conexión con datos laborales (SEPE) y trayectorias
formativas (TodoFP, Descubre la FP).

\section{PARTE 2. Programa Aula del Futuro adaptado a Orientación
Académico-Profesional}\label{parte-2.-programa-aula-del-futuro-adaptado-a-orientaciuxf3n-acaduxe9mico-profesional}

\subsection{2.1 Fundamentación
teórica}\label{fundamentaciuxf3n-teuxf3rica}

\begin{itemize}
\tightlist
\item
  \textbf{Modelo tipológico (Holland, Holland (1997)):} RIASEC para
  intereses y alineación con escenarios.
\item
  \textbf{Modelo de desarrollo (Super, Super (1990)):} Auto-concepto y
  etapas; 4.º ESO como exploración y cristalización inicial.
\item
  \textbf{Aprendizaje social (Krumboltz, Krumboltz (1979)):} Influencia
  de experiencias y factores contextuales (semi-rural, acceso
  tecnológico).
\item
  \textbf{Construcción de carrera (Savickas, Savickas (2013)):}
  Narrativas de propósito y adaptación a cambios futuros.
\item
  \textbf{Justificación:} Necesidad de apoyar decisiones post-ESO
  (Bachillerato vs FP vs otras vías) integrando autoconocimiento +
  análisis de oportunidades reales de entorno laboral y formación en
  Extremadura.
\end{itemize}

\subsection{2.2 Destinatarios}\label{destinatarios}

\begin{itemize}
\tightlist
\item
  Grupo de 4.º ESO (25--30 alumnos). Heterogeneidad en aspiraciones;
  parte con orientación hacia FP por contexto local.
\item
  Contexto semi-rural: menor exposición a sectores emergentes → se
  prioriza exploración ampliada.
\item
  Modalidad: Gran grupo con fases de trabajo cooperativo y microgrupos
  rotando por zonas AdF.
\item
  Adaptaciones: Fichas visuales simplificadas para alumnado con menor
  competencia lectora; apoyo digital gradual.
\end{itemize}

\subsection{2.3 Objetivos}\label{objetivos}

\textbf{General:} Facilitar la construcción inicial de un mapa de
carrera personalizado y viable a 3 años. \textbf{Específicos:} 1.
Identificar intereses y valores formativos (instrumento rápido RIASEC
adaptado). 2. Relacionar itinerarios post-ESO (Bachillerato modalidades,
FP familias) con perfil personal. 3. Analizar 2 sectores
emergentes/extremeños (agrotech, energías renovables) usando datos SEPE.
4. Elaborar un Canvas de Carrera (objetivo formativo + competencias +
pasos próximos). 5. Presentar y recibir feedback de pares para refinar
coherencia plan.

\subsection{2.4 Contenidos}\label{contenidos}

\begin{itemize}
\tightlist
\item
  Conceptuales: Itinerarios educativos; perfiles profesionales
  emergentes; competencias transversales (digital, colaboración,
  resolución).
\item
  Procedimentales: Búsqueda estructurada; uso de fichas de roles;
  construcción de Canvas; micro-pitch vocacional.
\item
  Actitudinales: Apertura, reflexión crítica, responsabilidad en toma de
  decisiones, aprendizaje permanente.
\item
  Secuenciación: Autoconocimiento → Exploración → Análisis laboral →
  Diseño → Presentación/feedback.
\end{itemize}

\subsection{2.5 Estructura modular (6
sesiones)}\label{estructura-modular-6-sesiones}

\begin{longtable}[]{@{}
  >{\raggedright\arraybackslash}p{(\linewidth - 8\tabcolsep) * \real{0.2000}}
  >{\raggedright\arraybackslash}p{(\linewidth - 8\tabcolsep) * \real{0.2000}}
  >{\raggedright\arraybackslash}p{(\linewidth - 8\tabcolsep) * \real{0.2000}}
  >{\raggedright\arraybackslash}p{(\linewidth - 8\tabcolsep) * \real{0.2000}}
  >{\raggedright\arraybackslash}p{(\linewidth - 8\tabcolsep) * \real{0.2000}}@{}}
\toprule\noalign{}
\begin{minipage}[b]{\linewidth}\raggedright
Sesión
\end{minipage} & \begin{minipage}[b]{\linewidth}\raggedright
Foco
\end{minipage} & \begin{minipage}[b]{\linewidth}\raggedright
Objetivos específicos
\end{minipage} & \begin{minipage}[b]{\linewidth}\raggedright
Espacio AdF principal
\end{minipage} & \begin{minipage}[b]{\linewidth}\raggedright
Producto
\end{minipage} \\
\midrule\noalign{}
\endhead
\bottomrule\noalign{}
\endlastfoot
1 & Autoconocimiento & 1 & Zona de investigación & Perfil intereses
resumido \\
2 & Opciones educativas & 2 & Zona de creación & Mapa itinerarios
personales \\
3 & Mundo laboral (sectores) & 3 & Zona interactiva & Fichas sectores
emergentes \\
4 & Diseño de mapa de carrera & 4 & Laboratorio diseño & Canvas
preliminar \\
5 & Ajuste y competencias & 4 & Zona presentación & Canvas final
revisado \\
6 & Pitch y feedback & 5 & Zona presentación & Mini pitch vocacional +
feedback \\
\end{longtable}

\subsection{2.6 Actividades clave (ejemplos por
sesión)}\label{actividades-clave-ejemplos-por-sesiuxf3n}

\begin{itemize}
\tightlist
\item
  Instrumento intereses (adaptación RIASEC breve; uso orientativo no
  diagnóstico).
\item
  Búsqueda guiada en TodoFP y Descubre la FP (matriz itinerarios).
\item
  Análisis de datos Observatorio SEPE (tendencias y perfiles).
\item
  Canvas de Carrera (plantilla estructurada: Objetivo formativo /
  Competencias / Recursos / Hitos próximos / Riesgos / Estrategias).
\item
  Pitch 60 segundos (problema, motivación, elección itinerario, primer
  paso).
\item
  Feedback cooperativo mediante tarjetas ``Potencia / Aclara / Añade''.
\end{itemize}

\subsection{2.7 Recursos y materiales}\label{recursos-y-materiales}

\begin{itemize}
\tightlist
\item
  Fichas impresas intereses y sectores.
\item
  Acceso web (TodoFP, SEPE).
\item
  Proyector / panel interactivo zona AdF.
\item
  Plantilla Canvas (A3).
\item
  Tarjetas feedback.
\end{itemize}

\subsection{2.8 Rol del orientador}\label{rol-del-orientador}

\begin{itemize}
\tightlist
\item
  Curador de recursos fiables.
\item
  Facilitador del análisis crítico (preguntas guía).
\item
  Moderador de feedback.
\item
  Registro observaciones de implicación y ajuste de apoyos.
\end{itemize}

\subsection{2.9 Evaluación del programa y del
alumnado}\label{evaluaciuxf3n-del-programa-y-del-alumnado}

\textbf{Matriz simple de evaluación (programa + alumnado):} \textbar{}
Criterio \textbar{} Indicador \textbar{} Instrumento \textbar{} Momento
\textbar{} Fuente \textbar{}
\textbar----------\textbar-----------\textbar-------------\textbar--------\textbar--------\textbar{}
\textbar{} Identificación intereses \textbar{} Perfil con ≥3 áreas
coherentes \textbar{} Ficha intereses \textbar{} Sesión 1 \textbar{}
Alumno \textbar{} \textbar{} Conocimiento itinerarios \textbar{}
Diferencia 2 opciones justificadas \textbar{} Mapa itinerarios
\textbar{} Sesión 2 \textbar{} Alumno \textbar{} \textbar{} Análisis
sectores \textbar{} Ficha con 2 datos empleo + 1 tendencia \textbar{}
Fichas sectores \textbar{} Sesión 3 \textbar{} Alumno \textbar{}
\textbar{} Coherencia Canvas \textbar{} Relación
intereses-itinerario-competencias \textbar{} Canvas carrera \textbar{}
Sesión 5 \textbar{} Orientador \textbar{} \textbar{} Comunicación
vocacional \textbar{} Pitch con estructura completa \textbar{}
Observación \textbar{} Sesión 6 \textbar{} Orientador/pares \textbar{}
\textbar{} Satisfacción programa \textbar{} ≥70\% valoración positiva
\textbar{} Cuestionario breve \textbar{} Sesión 6 \textbar{} Grupo
\textbar{} \textbar{} Reflexión personal \textbar{} 1 acción mejora
registrada \textbar{} Autoevaluación \textbar{} Sesión 6 \textbar{}
Alumno \textbar{}

Conforme a la Instrucción nº 14/2024 de organización y funcionamiento de
centros educativos de Extremadura (Secretaría General de Educación y
Formación Profesional - Junta de Extremadura (2024)), los registros de
evaluación de orientación se integrarán en la documentación de tutoría y
memoria final del Departamento de Orientación.

\section{PARTE 3. Aplicación simulada de sesión (Diseño de mapa de
carrera)}\label{parte-3.-aplicaciuxf3n-simulada-de-sesiuxf3n-diseuxf1o-de-mapa-de-carrera}

\textbf{Sesión seleccionada:} Sesión 4 (Diseño del mapa de carrera).
\textbf{Fecha simulada:} Semana del 05/02/2026.\\
\textbf{Duración:} 55 minutos.

\subsubsection{3.1 Objetivos de la
sesión}\label{objetivos-de-la-sesiuxf3n}

\begin{itemize}
\tightlist
\item
  Sintetizar autoconocimiento + opciones en un Canvas inicial.
\item
  Identificar 2 competencias a fortalecer y 2 hitos próximos (≤6 meses).
\end{itemize}

\subsubsection{3.2 Temporalización y
fases}\label{temporalizaciuxf3n-y-fases}

\begin{longtable}[]{@{}
  >{\raggedright\arraybackslash}p{(\linewidth - 8\tabcolsep) * \real{0.2000}}
  >{\raggedright\arraybackslash}p{(\linewidth - 8\tabcolsep) * \real{0.2000}}
  >{\raggedright\arraybackslash}p{(\linewidth - 8\tabcolsep) * \real{0.2000}}
  >{\raggedright\arraybackslash}p{(\linewidth - 8\tabcolsep) * \real{0.2000}}
  >{\raggedright\arraybackslash}p{(\linewidth - 8\tabcolsep) * \real{0.2000}}@{}}
\toprule\noalign{}
\begin{minipage}[b]{\linewidth}\raggedright
Min
\end{minipage} & \begin{minipage}[b]{\linewidth}\raggedright
Fase
\end{minipage} & \begin{minipage}[b]{\linewidth}\raggedright
Actividad
\end{minipage} & \begin{minipage}[b]{\linewidth}\raggedright
Método
\end{minipage} & \begin{minipage}[b]{\linewidth}\raggedright
Agrupamiento
\end{minipage} \\
\midrule\noalign{}
\endhead
\bottomrule\noalign{}
\endlastfoot
0--5 & Activación & Recordatorio intereses (lluvia breve) &
Participativa & Gran grupo \\
5--20 & Exploración guiada & Revisión fichas itinerarios y sectores &
Investigativa & Parejas \\
20--40 & Construcción & Completar Canvas Carrera & Colaborativa &
Microgrupos 3 \\
40--50 & Micro-pitch & Explicar elección itinerario y primer hito &
Experiencial & Microgrupos \\
50--55 & Autoevaluación & Tarjeta semáforo + nota de ajuste & Reflexiva
& Individual \\
\end{longtable}

\subsubsection{3.3 Recursos}\label{recursos}

\begin{itemize}
\tightlist
\item
  Plantilla Canvas (A3).
\item
  Fichas sectores (energías renovables / agrotech).
\item
  Acceso web básico (TodoFP, SEPE).
\end{itemize}

\subsubsection{3.4 Previsión de dificultades y
gestión}\label{previsiuxf3n-de-dificultades-y-gestiuxf3n}

\begin{longtable}[]{@{}
  >{\raggedright\arraybackslash}p{(\linewidth - 6\tabcolsep) * \real{0.2778}}
  >{\raggedright\arraybackslash}p{(\linewidth - 6\tabcolsep) * \real{0.2361}}
  >{\raggedright\arraybackslash}p{(\linewidth - 6\tabcolsep) * \real{0.2500}}
  >{\raggedright\arraybackslash}p{(\linewidth - 6\tabcolsep) * \real{0.2361}}@{}}
\toprule\noalign{}
\begin{minipage}[b]{\linewidth}\raggedright
Dificultad
\end{minipage} & \begin{minipage}[b]{\linewidth}\raggedright
Señal
\end{minipage} & \begin{minipage}[b]{\linewidth}\raggedright
Estrategia
\end{minipage} & \begin{minipage}[b]{\linewidth}\raggedright
Plan B
\end{minipage} \\
\midrule\noalign{}
\endhead
\bottomrule\noalign{}
\endlastfoot
Abstracción excesiva & Canvas vacío \textgreater10 min & Ejemplos modelo
& Tutor asigna ejemplo \\
Desigual participación & 1 alumno domina & Roles rotatorios & Pausa y
redistribución grupos \\
Falta tiempo pitch & \textless50\% presentan & Simplificar pitch a 30 s
& Trasladar cierre a siguiente sesión \\
Acceso limitado web & Lentitud conexión & Uso fichas impresas & Extender
exploración para refuerzo \\
\end{longtable}

\subsubsection{3.5 Evaluación prevista}\label{evaluaciuxf3n-prevista}

\begin{itemize}
\tightlist
\item
  Observación estructurada (lista cotejo).
\item
  Revisión Canvas (coherencia triádica
  intereses-itinerario-competencias).
\item
  Autoevaluación semáforo (Verde / Ámbar / Rojo) + acción mejora.
\end{itemize}

\section{PARTE 4. Reflexión final}\label{parte-4.-reflexiuxf3n-final}

\textbf{Importancia:} La orientación académico-profesional en 4.º ESO
facilita transiciones informadas y reduce decisiones reactivas.
\textbf{Rol del orientador:} Mediador entre información dispersa
(itinerarios, datos laborales, recursos digitales) y construcción
personal de sentido y propósito. \textbf{Fortalezas AdF:} Flexibilidad
espacial, favorece aprendizaje activo, integra tecnología y
colaboración. \textbf{Limitaciones:} Requiere formación docente en
diseño de escenarios; posible brecha digital en contextos semi-rurales;
tiempo curricular restringido. \textbf{Aprendizajes del proceso:} Diseño
modular mejora progresión cognitiva; la combinación de datos (SEPE) con
intereses aumenta relevancia percibida. \textbf{Mejoras propuestas:}
Incorporar seguimiento longitudinal (revisión Canvas trimestre),
entrevistas breves individuales, vincular evidencias del portfolio a
competencias marco europeo digital y personal-social. \textbf{Visión
futura:} Orientación integrada en proyectos interdisciplinares aumenta
empleabilidad y desarrollo de agencia formativa.

\section{Referencias}\label{referencias}

Las fuentes citadas se listan automáticamente vía BibLaTeX (APA) usando
el archivo consolidado \texttt{../practicum\_references.bib}.

\phantomsection\label{refs}
\begin{CSLReferences}{1}{1}
\bibitem[\citeproctext]{ref-holland1997}
Holland (1997). \emph{Making Vocational Choices: A Theory of Vocational
Personalities and Work Environments}. Psychological Assessment Resources

\bibitem[\citeproctext]{ref-krumboltz1979}
Krumboltz (1979). A Social Learning Theory of Career Decision Making.
\emph{Journal of Counseling Psychology}, \emph{25}. 1-9

\bibitem[\citeproctext]{ref-savickas2013}
Savickas (2013). Career Construction Theory and Practice. \emph{The
Career Development Quarterly}, \emph{61}(1). 54-63

\bibitem[\citeproctext]{ref-instruccion_14_2024_extremadura}
Secretaría General de Educación y Formación Profesional - Junta de
Extremadura (2024). Instrucción nº 14/2024, organización y
funcionamiento centros curso 2024-2025. DOE 26/06/2024.
https://\url{https://educarex.es/legislacion-sge/instrucciones.html}

\bibitem[\citeproctext]{ref-super1990}
Super (1990). \emph{A Life-Span, Life-Space Approach to Career
Development}. Jossey-Bass

\end{CSLReferences}




\end{document}
