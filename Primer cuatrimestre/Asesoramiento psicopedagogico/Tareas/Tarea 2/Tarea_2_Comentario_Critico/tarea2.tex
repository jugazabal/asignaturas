% Options for packages loaded elsewhere
% Options for packages loaded elsewhere
\PassOptionsToPackage{unicode}{hyperref}
\PassOptionsToPackage{hyphens}{url}
\PassOptionsToPackage{dvipsnames,svgnames,x11names}{xcolor}
%
\documentclass[
  spanish,
  12pt,
  a4paper,
]{article}
\usepackage{xcolor}
\usepackage[margin=2.5cm]{geometry}
\usepackage{amsmath,amssymb}
\setcounter{secnumdepth}{5}
\usepackage{iftex}
\ifPDFTeX
  \usepackage[T1]{fontenc}
  \usepackage[utf8]{inputenc}
  \usepackage{textcomp} % provide euro and other symbols
\else % if luatex or xetex
  \usepackage{unicode-math} % this also loads fontspec
  \defaultfontfeatures{Scale=MatchLowercase}
  \defaultfontfeatures[\rmfamily]{Ligatures=TeX,Scale=1}
\fi
\usepackage{lmodern}
\ifPDFTeX\else
  % xetex/luatex font selection
\fi
% Use upquote if available, for straight quotes in verbatim environments
\IfFileExists{upquote.sty}{\usepackage{upquote}}{}
\IfFileExists{microtype.sty}{% use microtype if available
  \usepackage[]{microtype}
  \UseMicrotypeSet[protrusion]{basicmath} % disable protrusion for tt fonts
}{}
\usepackage{setspace}
\makeatletter
\@ifundefined{KOMAClassName}{% if non-KOMA class
  \IfFileExists{parskip.sty}{%
    \usepackage{parskip}
  }{% else
    \setlength{\parindent}{0pt}
    \setlength{\parskip}{6pt plus 2pt minus 1pt}}
}{% if KOMA class
  \KOMAoptions{parskip=half}}
\makeatother
% Make \paragraph and \subparagraph free-standing
\makeatletter
\ifx\paragraph\undefined\else
  \let\oldparagraph\paragraph
  \renewcommand{\paragraph}{
    \@ifstar
      \xxxParagraphStar
      \xxxParagraphNoStar
  }
  \newcommand{\xxxParagraphStar}[1]{\oldparagraph*{#1}\mbox{}}
  \newcommand{\xxxParagraphNoStar}[1]{\oldparagraph{#1}\mbox{}}
\fi
\ifx\subparagraph\undefined\else
  \let\oldsubparagraph\subparagraph
  \renewcommand{\subparagraph}{
    \@ifstar
      \xxxSubParagraphStar
      \xxxSubParagraphNoStar
  }
  \newcommand{\xxxSubParagraphStar}[1]{\oldsubparagraph*{#1}\mbox{}}
  \newcommand{\xxxSubParagraphNoStar}[1]{\oldsubparagraph{#1}\mbox{}}
\fi
\makeatother


\usepackage{longtable,booktabs,array}
\usepackage{calc} % for calculating minipage widths
% Correct order of tables after \paragraph or \subparagraph
\usepackage{etoolbox}
\makeatletter
\patchcmd\longtable{\par}{\if@noskipsec\mbox{}\fi\par}{}{}
\makeatother
% Allow footnotes in longtable head/foot
\IfFileExists{footnotehyper.sty}{\usepackage{footnotehyper}}{\usepackage{footnote}}
\makesavenoteenv{longtable}
\usepackage{graphicx}
\makeatletter
\newsavebox\pandoc@box
\newcommand*\pandocbounded[1]{% scales image to fit in text height/width
  \sbox\pandoc@box{#1}%
  \Gscale@div\@tempa{\textheight}{\dimexpr\ht\pandoc@box+\dp\pandoc@box\relax}%
  \Gscale@div\@tempb{\linewidth}{\wd\pandoc@box}%
  \ifdim\@tempb\p@<\@tempa\p@\let\@tempa\@tempb\fi% select the smaller of both
  \ifdim\@tempa\p@<\p@\scalebox{\@tempa}{\usebox\pandoc@box}%
  \else\usebox{\pandoc@box}%
  \fi%
}
% Set default figure placement to htbp
\def\fps@figure{htbp}
\makeatother



\ifLuaTeX
\usepackage[bidi=basic]{babel}
\else
\usepackage[bidi=default]{babel}
\fi
% get rid of language-specific shorthands (see #6817):
\let\LanguageShortHands\languageshorthands
\def\languageshorthands#1{}


\setlength{\emergencystretch}{3em} % prevent overfull lines

\providecommand{\tightlist}{%
  \setlength{\itemsep}{0pt}\setlength{\parskip}{0pt}}



 
\usepackage[]{biblatex}
\addbibresource{references.bib}


\makeatletter
\@ifpackageloaded{caption}{}{\usepackage{caption}}
\AtBeginDocument{%
\ifdefined\contentsname
  \renewcommand*\contentsname{Tabla de contenidos}
\else
  \newcommand\contentsname{Tabla de contenidos}
\fi
\ifdefined\listfigurename
  \renewcommand*\listfigurename{Listado de Figuras}
\else
  \newcommand\listfigurename{Listado de Figuras}
\fi
\ifdefined\listtablename
  \renewcommand*\listtablename{Listado de Tablas}
\else
  \newcommand\listtablename{Listado de Tablas}
\fi
\ifdefined\figurename
  \renewcommand*\figurename{Figura}
\else
  \newcommand\figurename{Figura}
\fi
\ifdefined\tablename
  \renewcommand*\tablename{Tabla}
\else
  \newcommand\tablename{Tabla}
\fi
}
\@ifpackageloaded{float}{}{\usepackage{float}}
\floatstyle{ruled}
\@ifundefined{c@chapter}{\newfloat{codelisting}{h}{lop}}{\newfloat{codelisting}{h}{lop}[chapter]}
\floatname{codelisting}{Listado}
\newcommand*\listoflistings{\listof{codelisting}{Listado de Listados}}
\makeatother
\makeatletter
\makeatother
\makeatletter
\@ifpackageloaded{caption}{}{\usepackage{caption}}
\@ifpackageloaded{subcaption}{}{\usepackage{subcaption}}
\makeatother
\usepackage{bookmark}
\IfFileExists{xurl.sty}{\usepackage{xurl}}{} % add URL line breaks if available
\urlstyle{same}
\hypersetup{
  pdftitle={Comentario Crítico: Montanero (2014)},
  pdfauthor={Juan Marcos García Aranzábal},
  pdflang={es},
  colorlinks=true,
  linkcolor={blue},
  filecolor={Maroon},
  citecolor={Blue},
  urlcolor={Blue},
  pdfcreator={LaTeX via pandoc}}


\title{Comentario Crítico: Montanero (2014)}
\usepackage{etoolbox}
\makeatletter
\providecommand{\subtitle}[1]{% add subtitle to \maketitle
  \apptocmd{\@title}{\par {\large #1 \par}}{}{}
}
\makeatother
\subtitle{La investigación sobre el asesoramiento educativo en España:
una revisión de su metodología y resultados empíricos}
\author{Juan Marcos García Aranzábal}
\date{13 de noviembre de 2025}
\begin{document}
\maketitle


\setstretch{1.5}
\section{Resumen analítico}\label{resumen-analuxedtico}

\subsection{Objetivo principal del
artículo}\label{objetivo-principal-del-artuxedculo}

El artículo de Montanero (2014) presenta una revisión sistemática de los
trabajos empíricos publicados durante los últimos 20 años sobre el
asesoramiento educativo en el contexto del sistema educativo español. Su
propósito principal es analizar las metodologías empleadas en estas
investigaciones, sintetizar sus principales resultados y señalar sus
limitaciones metodológicas. El autor busca proporcionar una panorámica
del estado de la cuestión sobre la investigación del asesoramiento
psicopedagógico en España, identificando las principales tendencias,
enfoques y perspectivas que han caracterizado este campo de estudio.

\subsection{Principales hallazgos y
conclusiones}\label{principales-hallazgos-y-conclusiones}

Montanero identifica hallazgos convergentes en la investigación
revisada: (1) \textbf{Predominio del enfoque interpretativo} sobre
estudios experimentales de eficacia; (2) \textbf{Importancia y
dificultad} del asesoramiento, percibida tanto por profesores como
asesores; (3) \textbf{Discrepancia entre modelos}: los asesores aspiran
a promover desarrollo de centros pero dedican más tiempo a tareas
burocráticas; (4) \textbf{Documentación de estrategias} colaborativas
mediante análisis del discurso; y (5) \textbf{Atención a la diversidad}
como principal demanda de asesoramiento.

\subsection{Tipos de metodologías
revisadas}\label{tipos-de-metodologuxedas-revisadas}

Montanero clasifica las investigaciones en dos enfoques:

\textbf{1. Estudios centrados en explicar la eficacia} (relaciones
causales entre asesoramiento y resultados):

\begin{itemize}
\tightlist
\item
  \emph{Diseños experimentales/cuasi-experimentales}: Grupos control,
  pretest-postest, asignación aleatoria. Ejemplos: Sánchez y Rosales
  (1996) sobre habilidades discursivas; Montanero et al.~(2008) sobre
  explicaciones causales.
\item
  \emph{Serie temporal}: Múltiples mediciones durante el asesoramiento
  (García y Sánchez, 2007).
\item
  \emph{Estudios retrospectivos}: Datos institucionales para estudiar
  factores de eficacia escolar.
\end{itemize}

\textbf{2. Estudios centrados en comprender prácticas} (predominio
cualitativo):

\begin{itemize}
\tightlist
\item
  \emph{Encuestas cerradas}: Cuestionarios Likert sobre opiniones y
  percepciones (Marcelo, 1997; Boza, 2004).
\item
  \emph{Autoinformes cualitativos}: Grupos de discusión, entrevistas en
  profundidad, diarios (Escudero y Moreno, 1992; Arencibia y Moreno,
  2005).
\item
  \emph{Análisis del discurso}: Sistemas de categorización de sesiones
  de asesoramiento. Destacan los trabajos de Sánchez et al.~(1994-2008)
  analizando qué, cómo y para qué se comunica.
\end{itemize}

\section{Análisis metodológico}\label{anuxe1lisis-metodoluxf3gico}

\subsection{Evaluación de los enfoques
metodológicos}\label{evaluaciuxf3n-de-los-enfoques-metodoluxf3gicos}

Los diseños experimentales establecen relaciones causales mediante
medidas pre-post con grupos control e instrumentos validados, pero
presentan validez externa limitada por contextos controlados. Los
estudios interpretativos capturan la complejidad contextual y
concepciones subyacentes, documentando la distancia teoría-práctica,
aunque adolecen de escasa replicabilidad. Las principales limitaciones
incluyen: escasez de diseños experimentales rigurosos en educación
pre-universitaria; múltiples amenazas a la validez interna en estudios
retrospectivos; sesgos de deseabilidad social en encuestas; y falta de
triangulación metodológica en muchos estudios.

\subsection{Discusión sobre validez y
fiabilidad}\label{discusiuxf3n-sobre-validez-y-fiabilidad}

La validez y fiabilidad varían según el diseño: los experimentales
reportan fiabilidad aceptable (α = 0,89) y sistemas de categorías
validados, pero con validez externa limitada por muestras pequeñas. Las
encuestas establecen validez de constructo por expertos y pilotos,
aunque las tasas de respuesta (37\%-83\%) afectan la representatividad.
Los cualitativos emplean validación comunicativa y triangulación de
fuentes, pero los sistemas inductivos carecen frecuentemente de
fiabilidad inter-jueces, limitando la transferibilidad. Problemáticas
generales: múltiples amenazas a validez interna en diseños
retrospectivos, escasa replicabilidad cualitativa, y falta de estudios
longitudinales sobre cambio a largo plazo.

\subsection{Reflexión sobre métodos cualitativos
vs.~cuantitativos}\label{reflexiuxf3n-sobre-muxe9todos-cualitativos-vs.-cuantitativos}

El artículo evidencia predominio cualitativo/mixto sobre diseños
experimentales puros. Los métodos cualitativos ofrecen riqueza
contextual, acceso a concepciones implícitas, flexibilidad adaptativa y
relevancia ecológica al capturar el cómo y porqué de procesos complejos.
Sin embargo, presentan limitaciones: dificultad para establecer
causalidad, problemas de generalización, alta inversión temporal, riesgo
de sesgo interpretativo y menor consenso sobre rigor científico.

Los métodos cuantitativos/experimentales permiten relaciones causales
certeras, comparaciones sistemáticas, generalizabilidad y control de
variables extrañas, pero enfrentan dificultades en asesoramiento:
complejidad de manipular variables en contextos naturales, reduccionismo
que obvia aspectos procesuales, dilemas éticos en asignación aleatoria y
escasa viabilidad práctica.

Esta predominancia cualitativa refleja la naturaleza compleja,
contextualizada y colaborativa del asesoramiento español. Montanero
sugiere implícitamente mayor equilibrio: se requieren más estudios
experimentales que documenten eficacia, complementados con
investigaciones cualitativas que comprendan procesos subyacentes.

\section{Valoración crítica}\label{valoraciuxf3n-cruxedtica}

\subsection{Aportaciones del
artículo}\label{aportaciones-del-artuxedculo}

El artículo de Montanero aporta: (1) Primera revisión sistemática
comprehensiva de dos décadas de investigación española sobre
asesoramiento; (2) Marco clasificatorio claro distinguiendo estudios
orientados a explicar eficacia vs.~comprender prácticas; (3)
Documentación del predominio interpretativo-cualitativo; (4) Síntesis de
resultados empíricos convergentes sobre importancia/dificultad del
asesoramiento, modelos predominantes y demandas frecuentes; (5)
Visibilización de modelos metodológicamente rigurosos (ej. Sánchez et
al.); (6) Reconocimiento de la idiosincrasia del sistema español y
dificultades de transferir resultados internacionales; (7) Agenda
implícita señalando lagunas y necesidades de investigación futura.

\subsection{Limitaciones detectadas}\label{limitaciones-detectadas}

\textbf{Identificadas por Montanero:} Escasez de estudios experimentales
limita conclusiones sobre eficacia; amenazas a validez en diseños
retrospectivos ex-post-facto; dificultad de generalización de estudios
de caso idiográficos.

\textbf{Otras limitaciones:} Ausencia de evaluación sistemática de
calidad metodológica mediante criterios explícitos; falta de
meta-análisis para estimar tamaños del efecto agregados; criterios de
inclusión y bases de datos no explicitados, dificultando replicabilidad;
ausencia de análisis evolutivo de metodologías y temáticas durante las
dos décadas; limitada discusión de implicaciones prácticas
profesionales; probable sesgo de publicación al excluir tesis y reportes
institucionales no publicados.

\subsection{Propuestas de mejora y líneas
futuras}\label{propuestas-de-mejora-y-luxedneas-futuras}

\textbf{Metodológicas:} (1) Diseños mixtos secuenciales combinando
exploraciones cualitativas con evaluaciones experimentales; (2) Estudios
longitudinales documentando procesos de cambio a medio-largo plazo; (3)
Meta-análisis y revisiones sistemáticas con protocolos PRISMA; (4)
Estudios multinivel analizando variables individuales, grupales y
organizativas.

\textbf{Temáticas:} (5) Asesoramiento mediado por tecnología
virtual/híbrido post-COVID; (6) Competencias específicas del asesor
eficaz; (7) Estrategias para contextos inclusivos y atención a la
diversidad; (8) Modelos de formación inicial y continua de asesores; (9)
Perspectiva del alumnado sobre cambios derivados del asesoramiento; (10)
Estudios comparativos internacionales identificando elementos
transferibles y específicos.

\section{Referencias}\label{referencias}

Montanero, M. (2014). La investigación sobre el asesoramiento educativo
en España: una revisión de su metodología y resultados empíricos.
\emph{Revista Española de Pedagogía}, 72(259), 525-542.
https://reunir.unir.net/handle/123456789/3737


\printbibliography



\end{document}
